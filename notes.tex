\documentclass[12pt,titlepage]{article}
\usepackage[margin=1in]{geometry}

\begin{document}
  \begin{titlepage}
    \vspace*{\fill}
    \centering

    \textbf{\Huge BIOL 239 Course Notes} \\ [0.4em]
    \textbf{\Large Genetics} \\ [1em]
    \textbf{\Large Michael Socha} \\ [1em]
    \textbf{\large 4A Software Engineering} \\
    \textbf{\large University of Waterloo} \\
    \textbf{\large Spring 2018} \\
    \vspace*{\fill}
  \end{titlepage}

  \newpage 

  \tableofcontents

  \newpage

  \section{Course Overview}
    \subsection{Logistics}
      \begin{itemize}
        \item \textbf{Professor:} Christine Dupont
        \item \textbf{Email:} cdupont@uwaterloo.ca
        \item \textbf{Phone:} 888-4567 ext. 37804
        \item \textbf{Office:} B1 277
        \item \textbf{Office hours:} T 10:30-11:00am
      \end{itemize}

    \subsection{Summary of Topics}
      Topics covered in this course include:
      \begin{itemize}
        \item Transmission of traits
        \item Gene structure, function, and transmission
        \item Genes as carriers of information
        \item Gene regulation
        \item Population genetics
      \end{itemize}

    \section{Mendelian Genetics}
      \subsection{Artificial Selection}
        Artificial selection is the process in which humans select plants and animals to breed based 
        on particular traits. Artificial selection has been performed for tens of thousands of years,
        long before humans had a solid understanding of genetics. Examples of plants and animals domesticated
        through selective breeding include dogs, goats, sheep, cattle, cabbage, and maize.

      \subsection{Key Definitions}
        \begin{itemize}
          \item \textbf{Phenotype:} An observable characteristic in an individual, which can include an appearance of behaviour
          \item \textbf{Genotype:} The genetic composition of an individual, which in turn controls phenotypes
          \item \textbf{Gene:} A discreet unit of heredity
          \item \textbf{Allele:} An alternative form of a single gene
          \item \textbf{Dominant Allele:} An allele that is manifested in a phenotype regardless of the other allele.
          The phenotype in which this allele is manifested is known as the dominant phenotype.
          \item \textbf{Recessive Allele:} An allele that does not have an effect on a phenotype in when a dominant allele is present.
          The phenotype in which this allele is manifested is known as the recessive phenotype.
          \item \textbf{Parental Types:} Phenotypes that reflect a previously existing parental combination
          \item \textbf{Recombination Types:} Phenotypes that reflect a new combination of genes that occurs during gamete formation
          \item \textbf{Filal Generation:} Subsequent offspring in a sequence of breeding, starting with the parental (P) generation
        \end{itemize}

      \subsection{Mendel's Experiments}
        Mendel was a 19th century Augustinian monk that ran a series of experiments regarding trait inheritance. Much of the value of
        Mendel's experiments was due to how they featured:
        \begin{itemize}
          \item Using garden peas, which are easy to cross-fertizile, produce a large number of offspring, and have a short growing season
            (reproduce frequently). Garden peas also have clear-cut forms of the traits to test (e.g. yellow vs green, round vs wrinkled).
          \item Establishment of pure-breeding lines, which are lines are only breed with other members of the same genotype
          \item Carefully controlled breeding (use of reciprocal crosses and self-fertilization)
        \end{itemize}

      \subsection{Genetics Prior to Mendel}
        Multiple incorrect theories existed prior to Mendel's laws of inheritance. These included:
        \begin{itemize}
          \item One parent contributing more to an offspring the other - disproved through reciprocal crosses
          \item Parental traits become blended (i.e. no discrete traits) - disproved through reappearance of reciprocal traits
        \end{itemize}

      \subsection{Mendel's Laws}
        Through his experimentation, Mendel devised the following laws of inheritance:

        \subsubsection{Law of Segregation}
          This law states that each trait is controlled by two alleles which separate upon gamete formation and recombine when a zygote is
          formed.

        \subsubsection{Law of Independent Assortment}
          This law states that alleles for separate phenotypic traits are transmitted to offspring independent of one another. This can be demonstrated
          through dihybrid experiments, where individuals differing in multiple traits are crossed with one another.

        \subsubsection{Law of Dominance}
          This law introduces the concept of dominant and recessive alleles, and states that recessive alleles do not affect the phenotype unless
          both alleles are recessive.

      \subsection{Types of Crosses}
        \subsubsection{Monohybrid crosses}
          Monohybrid crosses are mating between individuals that only differ in one trait. The F1 filal features only the dominant phenotype, while
          the F2 filal features the dominate phenotype and recessive phenotype in a 3:1 ratio.
          such crosses.

        \subsubsection{Dihybrid crosses}
          Dihybrid crosses are mating between individuals that only differ in two unrelated traits. The F1 filal features both dominant phenotypes, while
          the F2 filal features a 9:3:3:1 ratio of traits 1 and 2 both being dominant, trait 1 being dominant and trait 2 being recessive,
          trait 1 being recessive and trait 2 being dominant, and both traits being recessive.

        \subsubsection{Multihybrid crosses}
          Dihybrid crosses are mating between individuals that only differ in three or more unrelated traits.

      \subsection{Analysis Techniques}
        \subsubsection{Punnet Squares}
          A Punnet Square is a chart that lists the possible gametes that may combine in a cross (one gamete list takes up 1 dimension), and lists all
          possible resulting combinations.

        \subsubsection{Branched-line Diagrams}
          A branched-line diagram features one column for each gene in a cross, with the resulting phenotypes shown at the end. This type of diagram
          is useful for multihybrid crosses, where Punnet Squares become hard to read.

        \subsubsection{Probability Rules}
          Probability rules (i.e. product rule and sum rule) can be also be used to analyze the results of crosses, often in combination with the above
          techniques.

    \section{Extensions to Mendel for Single-Gene Inheritance}
      \subsection{Incomplete Dominance}
        Incomplete dominance describes the situation where F1 offspring have a phenotype that resembles neither of their parents, but rather appears as a blend
        of the two. An example is crossing red and white flowers to generate the F1 generation, and the F1 generation's flowers being pink.

      \subsection{Codominance}
        Codominance described the situation where F1 offspring have a phenotype that combined features from both parents. An example would be crossing dotted and
        spotted lentils to generate the F1 generation, and the F1 generation being both spotted and dotted.

      \subsection{Genes with more than 2 Alleles}
        Some genes can have more than 2 alleles. In this case, filal genotype and phenotype ratios can be much different from those experiments with only
        2 alleles.

      \subsection{Gene Mutations}
        Mutations of genes occur in nature at a fairly low frequency; the frequency of gametes with a mutations is typically between 1 in 10000 to 1 in 1000000,
        depending on the gene. Such mutations are allow for the creation of new alleles.

      \subsection{Allele Frequencies}
        The percentage of a particular allele in a gene's entire population is knows as its allele frequency. Alleles that have a relatively high frequency
        (typically over 1\%) are known as wild-type alleles, while other alleles are known as mutant alleles. A gene with only one wild-type allele is known
        as monomorphic, while genes with multiple wild-type alleles are known as polymorphic.

      \subsection{Pleiotropy}
        Pleiotropy is the concept of a single gene determining multiple (often seemingly unrelated) phenotypic traits. Mendel himself observed this phenomenon
        during his experiments, where pea seed coat color matched to flower color, implying that there was a common control for both traits. Note that a single
        allele may be dominant with respect to some traits and recessive with respect to others, so dominant and recessive for such alleles must be defined
        with the context of a specific phenotype.

      \subsection{Recessive Lethal Alleles}
        Some alleles that are harmless when combined with a different allele can render an individual inviable when there are two copies of that allele. This
        allele is recessive in terms of lethality, but may be dominant with respect to some other phenotype. In this case, the phenotypic ratios of this dominant
        trait in the F1 generation would be 2:1 instead of 3:1, since individuals with two copies of this trait's dominant allele cannot survive. Note that some
        recessive lethal alleles may cause delayed lethality, in which case the phenotypic ratios would not be 2:1, since affected individuals may survive for
        some time (and may even reproduce).

    \section{Extensions to Mendel for Gene Interactions}
      Traits in an organism that arise from the actions of multiple genes are known as polygenic, while traits that are controlled by multiple genes and the environment
      are known as multifactorial. These two types of traits are knows as complex traits, and account for the majority of traits in organisms. Genotypic classes are
      groupings of related genotypes that produce a particular phenotype.

      \subsection{Complementary Gene Action}
        Complementary gene action refers to multiple genes working together to produce a particular trait. As an example, consider a combination of 2 genes with alleles
        Aa and Bb, and assume that a certain trait only surfaces in the A-B- phenotype. In the F2 population, this will result in a 9:7 dominant to non-dominant trait
        ratio.

      \subsection{Epistasis}
        Epistasis refers to one allele masking the effect of another allele. The allele performing the masking is known as the epistatic allele.

        A situation where two recessive alleles are required to mask the effect of some other allele is known as recessive epistasis. In a recessive
        epistasis example where a BBEE generation is crossed with a bbee generation, the F2 ratios are 9:3:4 (9 parts B-E-), 3 parts bbE-, and 4 parts --ee).

        A situation where one dominant allele is sufficient to mask the effect of some other allele is known as dominant epistasis. In a dominant
        epistasis example where a BBEE generation is crossed with a bbee generation, the F2 ratios are 12:3:1 (12 parts B---), 3 parts bbE-, and 1 part bbee).
        Should a bbee phenotype be the same as a B--- phenotype, this ratio can be simplified to 13:3 - this case is known as dominant suppression.

      \subsection{Redundant Genes}
        Redundant genes are two genes that control a very similar phenotype. As an example, if genes A and B are redundant and a AABB generation is crossed
        with a aabb generation, the F2 ratios are 15:1.

      \subsection{Heterogeneous Traits and Complementation}
        Phenotypic traits that can arise as a result from multiple different genes are known as heterogeneous. Thus, it is possible that two individuals
        sharing the same traits have differing genetic causes of those traits. If these individuals have offspring which exhibit a wild-type phenotype, then
        complementation has occurred, meaning that different genes controlled the mutant phenotype for both parents, and that they were both recessive. Thus,
        each parent could apply a dominant allele to complement the other parent's recessive mutant allele.

      \subsection{Penetrance and Expressivity}
        A phenotype may depend on more factors than just the underlying genotype. These factors may include environmental factors, modifier genes, and random
        chance. Penetrance is used to describe how many members of a population with a particular genotype show the expected phenotype. Penetrance can be complete
        (i.e. 100\%) or incomplete. Expressivity refers to the intensity with which a particular genotype is expressed in a phenotype, which may be variable or
        unvarying.

      \subsection{Modifier Genes}
        Genes that alter the phenotype produced by alleles of other genes are known as modifier genes.

    \section{Pedigrees}
      \subsection{Single-Gene Traits in Humans}
        Most human traits are controlled by the interaction of many genes. In fact, most single-gene traits in humans evoke life-threatening diseases, such as
        cystic fibrosis and Tay-Sachs disease.

      \subsection{Studying Hereditary Traits in Humans}
        Humans make poor genetic testing subjects; on top of many ethical issues, humans tend to not have many offspring, do not interbreed much
        (i.e. no F2 generations), and rarely choose mates on purely genetic grounds. Human genetic lineage is often analyzed using pedigrees, which
        are diagrams detailing a family's genetic features over several generations.

      \subsection{Pedigree Notation}
        \begin{itemize}
          \item \textbf{Sex:}
          \begin{itemize}
            \item \textbf{Male:} Square
            \item \textbf{Female:} Circle
            \item \textbf{Unspecified:} Diamond
          \end{itemize}

          \item \textbf{Trait infliction:}
            \begin{itemize}
              \item \textbf{Trait inflicted:} Filled shape
              \item \textbf{Trait not inflicted:} Unfilled shape
            \end{itemize}

          \item \textbf{Mating:}
            \begin{itemize}
              \item \textbf{Consaguineous mating:} Double horizontal line
              \item \textbf{Non-Consaguineous mating:} Single horizontal line
            \end{itemize}
        \end{itemize}

      \subsection{Patterns of Inheritance}
        Dominant traits tend to follow a vertical pattern of inheritance, being transferred over multiple generations. On the other hand,
        recessive traits tend to follow a horizontal pattern of inheritance, where in order for offspring to exhibit a recessive phenotype,
        both parents must carry the allele that causes that phenotype.

    \section{Chromosomes}
      \subsection{Experimentation History}
        In the 17th century, semen was determined to contain spermatozoa, which was hypothesized to be able to enter an egg and achieve fertilization.
        This hypothesis was confirmed by further experimentation in the 19th century. Around 1880, threadlike structured later confirmed to be
        chromosomes were observed during cell division. In the early 1900s, a link was established between chromosomes and to Mendelian principles of
        segregation and independent assortment, giving rise to the chromosomal theory of inheritance.

      \subsection{What are Chromosomes?}
        Chromosomes are organelles that package DNA. Most cells that compose an individual, as well as zygotes, carry two matching sets of chromosomes.
        These cells are knows as diploids. Chromosomes in a pair that contain the same genes are called homologous chromosomes (or homologues).
        Since one chromosome in each pair comes from each parent, homologues may contain different alleles. Cells that only contain one set of chromosomes,
        such as gametes, as known as haploids.

        A karyotype is a visual description of the complete set of chromosomes found in the cell of an organism. In a karyotype, homologous pairs are shown
        together ordered by decreasing size. The matching pairs of chromosomes are known as autosomes, which are distinct from sex chromosomes.

      \subsection{Sex Chromosomes and Sex Determination}
        Sex chromosomes determine the sex of an organism. In humans, the 23rd pair of chromosomes are sex chromosomes, where females have an XX pair and males have
        an non-homologous XY pair. The Y chromosome's SRY (sex-determining region of Y) gene is specifically what determines maleness. Sex-determining mechanisms
        can vary significantly between different species.

    \section{Mitosis}
      Mitosis is a process of cell division in which a cell produces two genetically identical daughter cells. In the cell cycle, mitosis is known as the M phase.

      \subsection{Interphase}
        Interphase is the part of the cell cycle that occurs between divisions. Interphase consists of the gap 1 (G1), synthesis (S), and gap 2 (G2) phases.

        \subsubsection{G1}
          G1 lasts from the birth of a new cell until chromosome replication. In this phase, most of a cell's growth occurs. The duration of G1 can vary greatly,
          from between a few hours in a newly developing embryo, to the remaining of an individual's lifetime for mature brain cells.

        \subsubsection{S}
          In the S phase, the cell duplicates its genetic material by synthesizing DNA. Each chromosome doubles to produce identical sister chromatids, which become
          visible once chromosomes condense at the beginning of mitosis. These sister chromatids remain joined to each other at the centromere until mitosis begins.

        \subsubsection{G2}
          During G2, the cell may grow, though typically less than during G1. The cell also synthesized proteins necessary for mitosis.

        Also, during interphase, fine microtubules known as centrioles radiate out into a cell's cytoplasm from a structure known as a centrosome. Centrosomes are
        typically composed of a pair of structures known as centrioles.

      \subsection{Phases of Mitosis}

        \subsubsection{Prophase}
          During interphase, the cell nucleus remains intact, and chromosomes are masses of chromatin, which looks like a mass of fine tangled string. At prophase,
          chromosomes condense, move apart toward opposite poles of a cell. Also, nucleoli begin to disappear, and microtubules disappear and are replaced by a
          set of dynamic microtubules that grow from and shrink toward their centrosomal organizing centers. The centrosomes also move apart to opposite poles of the
          cell.

        \subsubsection{Prometeaphase}
          In prometaphase, the nuclear envelope breaks down, and microtubules from the centrosomes invade the nuclues. Sister chromatids attach to microtubules from
          opposite centrosomes through the kinetochore, a structure in the centromere region of each chromatid specialized for transport.

        \subsubsection{Metaphase}
          In metaphase, chromosomes align on the metaphase plate, which is an equator halfway between the cell's two poles. In this phase, forces pulling sister
          chromatids toward opposite poles are in balance with tension across the chromosomes.

        \subsubsection{Anaphase}
          During anaphase, the centromeric connections are severed, and separated sister chromatids begin to move to opposite cell poles to which they are connected
          through kinetochere microtubles. This separation of sister chromatids to pass genetically identical material onto both daughter cells is known as disjunction.

        \subsubsection{Telophase}
          During telophase, the nuclear member and nucleoi reform, the spindle fibres disappear, and chromosomes untangle to become a mass of chromatin. This phase can be
          seen as somewhat of a reverse of prophase.

        \subsubsection{Cytokinesis}
          In cytokenesis, the nuclei emerging at each end of the elongated parent cell as packaged into two separate daughter cells. Processes of cytokenesis may actually
          begin during anaphase, though do not complete until after telophase.

    \section{Meiosis}
      The majority of cells that make up an individual are somatic cells, which either mitotically divide or are arrested in a G phase of the cell cycle. On the other hand,
      Germ cells are incorporated into reproductive organs, where they eventually undergo meiosis, a type of cell division that yields 4 gamete haploid cells from a single
      diploid cell. Meiosis contains two separate cell divisions, known as meiosis I and meiosis II.

      \subsection{Meiosis I}

        \subsubsection{Prophase I}
          Lepotene is the first substage of prophase I, and is the phase in which chromosomes start to thicken. These chromosomes have already been priorly duplicated.

          In Zygotene, each chromosomes seeks out its homologous partner, which become zipped together in a process known as synapsis. Each synapses chromosome pair is known
          as a bivalent (contains two chromosomes), or a tetrad (contains four chromatin). One side of a bivalent has maternal genetic material, and the other has paternal
          genetic material.

          Pachytene begins after synapsis, and consists of structured called recombination nodules exchanging parts of maternal and paternal chromatids of homologous chromosomes.
          this exchange is known as recombination or crossing-over.

          In diplotene, the synaptonemal zipper complex begins to disappear, and the tetrads of four chromatin become visible and start to move apart. However, homologous
          chromosomes remain tightly connected at areas called chiasmata, which represent sites where recombination occurred.

          In diakinesis, the chromatids condense and shorten even further. Also, the nuclear membrane begins to break down, and spindles begin to attach to kinetochere structures.

        \subsubsection{Metaphase I}
          During metaphase I, the tetrads formed in prophase I line up along the metaphase plate. These tetrads are attached to spindles on opposite poles, and begin to separate.

        \subsubsection{Anaphase I}
          In anaphase I, the chiasmata joining homologues are dissolved, allowing the chromosomes to separate and move toward opposite poles.

        \subsubsection{Telophase I}
          Telophase I begins when nuclear membranes start to form around the chromosomes that have moved to each pole. Each daughter nuclei contains half the number of chromosomes
          in the original parent (meiosis I is known as reducational division), but there are two sister chromatids for each chromosome.

      \subsection{Meiosis II}
        Following a brief interphase following meiosis I known as interkinesis, Meiosis II is performed on both daughter cells. Meiosis II is a very similar process to mitosis,
        and yield four haploid cells.

      \subsection{Mistakes in Meiosis}
        Segregational errors during meiotic divisions can lead to aberrations. An example cause is when homologues fail to separate during meiosis I, causing a trisomy through
        a process known as nondisjuction. Many trisomies are lethal in utero, though others, such as trisomy 21, are not.

        On a separate note, some hybrid animals carry non-homologous chromosomes that cannot pair up and segregate during meiosis. Viable gametes cannot form in such animals,
        so they are sterile.

      \subsection{Meiosis and Genetic Diversity}
        Two main factors contribute to genetic diversity in meiosis. The first is that the pole maternal or paternal homologues migrate to is random, so the homologue pairs
        are able to assort independently of one another. The second factor is the resuffling of genetic material that occurs during recombination.

    \section{Gametogenesis}
      Embryonic germ cells (collectively known as a germ line) are specialized diploid cells that can give rise to haploid cells through meiosis. These meiotic divisions lead
      to the production of gametes, with either the resulting haploids being gametes, or some of the resulting haploids undergoing some sort of differentiation to become
      gametes.

      \subsection{Oogenesis}
        Oogenesis is the process that produces egg cells, also known as ovums. Oocytes are cells that undergo meiotic divisions to ultimately generate on ovum. About half
        a million primary oocytes are stored in the ovaries, and are arrested in the diplotene substage of prophase I. During ovulation, a primary oocyte is released and
        produces a secondary oocyte and a far smaller polar body. The first polar body usually arrests it development, while the secondary oocyte proceeds as far as the
        metaphase of meiosis II. Upon fertilization, meiosis II completes, with the secondary oocyte producing a mature ovum and a second polar body. The long interval
        between to beginning and completion of meiosis for oocytes may lead to segregational errors such as trisomies for older females.

      \subsection{Spermatogenesis}
        Spermatogenesis is the process that produces sperm cells. Spermatocytes are cells that undergo meiotic divisions to produce sperm cells. Unlike oocytes,
        spermatocytes undergo a symmetric meiosis I and II, yielding four spermatids for each spermatocyte. Also, spermatogenesis can continue throughout a male's lifetime.

    \section{Validation of Chromosome Theory}

      \subsection{Key Claims}
        The key claims of the chromosomal theory of inheritance are that:
        \begin{enumerate}
          \item Hereditary information is stored on genes, which are located in chromosomes.
          \item The egg and sperm contribute equally to the genetic composition of the resulting zygote.
        \end{enumerate}

      \subsection{Overview of Evidence}
        Based on already covered material, the following points support the chromosomal theory of inheritance:
        \begin{enumerate}
          \item The number of chromosomes in somatic cells is constant.
          \item Sexual identity is associated with particular chromosomes.
        \end{enumerate}

        The following evidence would strengthen the theory's claims:
        \begin{enumerate}
          \item The inheritance of genes corresponding to the inheritance of chromosomes
          \item The transmission of particular chromosomes corresponding with the transmission of particular traits (i.e. traits other than sex)
        \end{enumerate}

        \subsection{Experimental Results}
          Key experiments were run by American biologist Thomas Hunt Morgan on a type of fly called Drosophilia. These experiments ultimately
          confirmed that a gene controlling eye color resides on an X chromosome, which is also used for sex determination (this makes the eye
          color an X-linked trait). These results were obtained by observing the phenotypic ratios of eye color between males and females, as
          well as analyzing individuals that had undergone non-disjunction to confirm that eye color was ultimately controlled by genes on X
          chromosomes rather than an individual's sex.

    \section{Aneuploidy}
      Aneuploidy refers to an abnormal number of chromosomes in a cell (i.e. not an exact multiple of haploid number $n$). Individuals lacking a
      chromosome in a pair are known as monosomic, while individuals with an extra chromosome are known as trisomic.

      \subsection{Autosomal Aneuploidy}
        Autosomal aneuploidy tends to be lethal for most chromosomes. However, there are some exceptions, such as trisomy 21 in humans, which
        causes Down syndrome. As a general rule, autosomal trisomies can sometimes be viable, while autosomal monosomies cannot.

      \subsection{Sex Chromosome Aneuploidy}
        Aneuploidies in sex chromosomes tend to be more viable than in autosomal chromosomes. For aneuploidies of the X chromosome in particular,
        X-chromosome inactivation represses the expression of most genes on all but one X chromosome. However, X-chromosome aneuploidies commonly
        cause infertility, since X-chromosome reactivation occurs in the oogonia, which is normally necessary in order to ensure that all of
        the produced ovums can support development after fertilization (as opposed to half with only X-chromosome inactivation). X-chromosome
        reactivation also occurs in males with multiple X chromosomes, typically preventing them from producing sperm.

      \subsection{Meiotic Nondisjunction}
        Mistakes in chromosome segregation during meiosis can cause aneuploidies. Nondisjunction during meiosis I produces four cells with
        aneuploidies, while nondisjunction during meiosis II produces only two such cells. Studies suggest that meiotic nondisjunction is may
        result from problems during recombination. In particular, a lack of chiasmata formation removes the mechanism for ensuring that
        maternal and paternal chromosomes go to opposite poles during anaphase I.

      \subsection{Mitotic Nondisjunction}
        Aneuploidies may also result from the failure of sister chromatids to separate during mitosis, producing trisomic and monosomic daughter
        cells. Other mistakes may also occur, such as chromosome loss, which produces one monosomic and one diploid cell. Aneuploid cells may
        survive and undergo further rounds of mitosis, potentially producing a mosaic organism with differing chromosome numbers in across
        different tissue. One such example is with gynandromorphic organisms, which are organisms that contain both male and female characteristics.
        Another example is with Turner syndrome (XO); most affected individuals start off as XX zygotes, but lose an X chromosome early during embryonic
        development.

    \section{Euploidy}
      Euploid cells contain only complete sets of chromosomes, but might be monoploids (containing only 1 set of chromosomes) or polyploids
      (containing 3 or more sets of chromosomes).

      \subsection{Monoploidy}
        Monoploid organisms tend to develop by parthenogenesis, where an ovum develops without having been fertilized. Due to the mechanics of meiosis,
        monoploids are typically infertile. One potential benefit of monoploidy, often used by plant breeders, is that desirable recessive traits appear
        since there is no matching (potentially dominant) allele.

      \subsection{Triploidy}
        Triploids can result from the union of a gamete and haploid cells, and are almost always sterile. This sterility is often used by plan breeders,
        who may want to deliberately grow seedless fruit.

\end{document}
