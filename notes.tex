\documentclass[12pt,titlepage]{article}
\usepackage[margin=1in]{geometry}

\begin{document}
  \begin{titlepage}
    \vspace*{\fill}
    \centering

    \textbf{\Huge BIOL 239 Course Notes} \\ [0.4em]
    \textbf{\Large Genetics} \\ [1em]
    \textbf{\Large Michael Socha} \\ [1em]
    \textbf{\large 4A Software Engineering} \\
    \textbf{\large University of Waterloo} \\
    \textbf{\large Spring 2018} \\
    \vspace*{\fill}
  \end{titlepage}

  \newpage 

  \tableofcontents

  \newpage

  \section{Course Overview}
    \subsection{Logistics}
      \begin{itemize}
        \item \textbf{Professor:} Christine Dupont
        \item \textbf{Email:} cdupont@uwaterloo.ca
        \item \textbf{Phone:} 888-4567 ext. 37804
        \item \textbf{Office:} B1 277
        \item \textbf{Office hours:} T 10:30-11:00am
      \end{itemize}

    \subsection{Summary of Topics}
      Topics covered in this course include:
      \begin{itemize}
        \item Transmission of traits
        \item Gene structure, function, and transmission
        \item Genes as carriers of information
        \item Gene regulation
        \item Population genetics
      \end{itemize}

    \section{Mendelian Genetics}
      \subsection{Artificial Selection}
        Artificial selection is the process in which humans select plants and animals to breed based 
        on particular traits. Artificial selection has been performed for tens of thousands of years,
        long before humans had a solid understanding of genetics. Examples of plants and animals domesticated
        through selective breeding include dogs, goats, sheep, cattle, cabbage, and maize.

      \subsection{Key Definitions}
        \begin{itemize}
          \item \textbf{Phenotype:} An observable characteristic in an individual, which can include an appearance of behaviour
          \item \textbf{Genotype:} The genetic composition of an individual, which in turn controls phenotypes
          \item \textbf{Gene:} A discreet unit of heredity
          \item \textbf{Allele:} An alternative form of a single gene
          \item \textbf{Dominant Allele:} An allele that is manifested in a phenotype regardless of the other allele.
          The phenotype in which this allele is manifested is known as the dominant phenotype.
          \item \textbf{Recessive Allele:} An allele that does not have an effect on a phenotype in when a dominant allele is present.
          The phenotype in which this allele is manifested is known as the recessive phenotype.
          \item \textbf{Parental Types:} Phenotypes that reflect a previously existing parental combination
          \item \textbf{Recombination Types:} Phenotypes that reflect a new combination of genes that occurs during gamete formation
          \item \textbf{Filal Generate:} Subsequent offspring in a sequence of breeding, starting with the parental (P) generation
        \end{itemize}

      \subsection{Mendel's Experiments}
        Mendel was a 19th century Augustinian monk that ran a series of experiments regarding trait inheritance. Much of the value of
        Mendel's experiments was due to how they featured:
        \begin{itemize}
          \item Using garden peas, which are easy to cross-fertizile, produce a large number of offspring, and have a short growing season
            (reproduce frequently). Garden peas also have clear-cut forms of the traits to test (e.g. yellow vs green, round vs wrinkled).
          \item Establishment of pure-breeding lines, which are lines are only breed with other members of the same genotype
          \item Carefully controlled breeding (use of reciprocal crosses and self-fertilization)
        \end{itemize}

      \subsection{Genetics Prior to Mendel}
        Multiple incorrect theories existed prior to Mendel's laws of inheritance. These included:
        \begin{itemize}
          \item One parent contributing more to an offspring the other - disproved through reciprocal crosses
          \item Parental traits become blended (i.e. no discrete traits) - disproved through reappearance of reciprocal traits
        \end{itemize}

      \subsection{Mendel's Laws}
        Through his experimentation, Mendel devised the following laws of inheritance:

        \subsubsection{Law of Segregation}
          This law states that each trait is controlled by two alleles which separate upon gamete formation and recombine when a zygote is
          formed.

        \subsubsection{Law of Independent Assortment}
          This law states that alleles for separate phenotypic traits are transmitted to offspring independent of one another. This can be demonstrated
          through dihybrid experiments, where individuals differing in multiple traits are crossed with one another.

        \subsubsection{Law of Dominance}
          This law introduces the concept of dominant and recessive alleles, and states that recessive alleles do not affect the phenotype unless
          both alleles are recessive.

      \subsection{Types of Crosses}
        \subsubsection{Monohybrid crosses}
          Monohybrid crosses are mating between individuals that only differ in one trait. The F1 filal features only the dominant phenotype, while
          the F2 filal features the dominate phenotype and recessive phenotype in a 3:1 ratio.
          such crosses.

        \subsubsection{Dihybrid crosses}
          Dihybrid crosses are mating between individuals that only differ in two unrelated traits. The F1 filal features both dominant phenotypes, while
          the F2 filal features a 9:3:3:1 ratio of traits 1 and 2 both being dominant, trait 1 being dominant and trait 2 being recessive,
          trait 1 being recessive and trait 2 being dominant, and both traits being recessive.

        \subsubsection{Multihybrid crosses}
          Dihybrid crosses are mating between individuals that only differ in three or more unrelated traits.

      \subsection{Analysis Techniques}
        \subsubsection{Punnet Squares}
          A Punnet Square is a chart that lists the possible gametes that may combine in a cross (one gamete list takes up 1 dimension), and lists all
          possible resulting combinations.

        \subsubsection{Branched-line Diagrams}
          A branched-line diagram features one column for each gene in a cross, with the resulting phenotypes shown at the end. This type of diagram
          is useful for multihybrid crosses, where Punnet Squares become hard to read.

        \subsubsection{Probability Rules}
          Probability rules (i.e. product rule and sum rule) can be also be used to analyze the results of crosses, often in combination with the above
          techniques.

      \subsection{Extensions to Mendel for Single-Gene Inheritance}
        \subsubsection{Incomplete Dominance}
          Incomplete dominance describes the situation where F1 offspring have a phenotype that resembles neither of their parents, but rather appears as a blend
          of the two. An example is crossing red and white flowers to generate the F1 generation, and the F1 generation's flowers being pink.

        \subsubsection{Codominance}
          Codominance described the situation where F1 offspring have a phenotype that combined features from both parents. An example would be crossing dotted and
          spotted lentils to generate the F1 generation, and the F1 generation being both spotted and dotted.

        \subsubsection{Genes with more than 2 Alleles}
          Some genes can have more than 2 alleles. In this case, filal genotype and phenotype ratios can be much different from those experiments with only
          2 alleles.

        \subsubsection{Gene Mutations}
          Mutations of genes occur in nature at a fairly low frequency; the frequency of gametes with a mutations is typically between 1 in 10000 to 1 in 1000000,
          depending on the gene. Such mutations are allow for the creation of new alleles.

        \subsubsection{Allele Frequencies}
          The percentage of a particular allele in a gene's entire population is knows as its allele frequency. Alleles that have a relatively high frequency
          (typically over 1\%) are known as wild-type alleles, while other alleles are known as mutant alleles. A gene with only one wild-type allele is known
          as monomorphic, while genes with multiple wild-type alleles are known as polymorphic.

        \subsubsection{Pleiotropy}
          Pleiotropy is the concept of a single gene determining multiple (often seemingly unrelated) phenotypic traits. Mendel himself observed this phenomenon
          during his experiments, where pea seed coat color matched to flower color, implying that there was a common control for both traits. Note that a single
          allele may be dominant with respect to some traits and recessive with respect to others, so dominant and recessive for such alleles must be defined
          with the context of a specific phenotype.

        \subsubsection{Recessive Lethal Alleles}
          Some alleles that are harmless when combined with a different allele can render an individual inviable when there are two copies of that allele. This
          allele is recessive in terms of lethality, but may be dominant with respect to some other phenotype. In this case, the phenotypic ratios of this dominant
          trait in the F1 generation would be 2:1 instead of 3:1, since individuals with two copies of this trait's dominant allele cannot survive. Note that some
          recessive lethal alleles may cause delayed lethality, in which case the phenotypic ratios would not be 2:1, since affected individuals may survive for
          some time (and may even reproduce).

      \subsection{Extensions to Mendel for Gene Interactions}
        Traits in an organism that arise from the actions of multiple genes are known as polygenic, while traits that are controlled by multiple genes and the environment
        are known as multifactorial. These two types of traits are knows as complex traits, and account for the majority of traits in organisms. Genotypic classes are
        groupings of related genotypes that produce a particular phenotype.

        \subsubsection{Complementary Gene Action}
          Complementary gene action refers to multiple genes working together to produce a particular trait. As an example, consider a combination of 2 genes with alleles
          Aa and Bb, and assume that a certain trait only surfaces in the A-B- phenotype. In the F2 population, this will result in a 9:7 dominant to non-dominant trait
          ratio.

        \subsubsection{Epistasis}
          Epistasis refers to one allele masking the effect of another allele. The allele performing the masking is known as the epistatic allele.

          A situation where two recessive alleles are required to mask the effect of some other allele is known as recessive epistasis. In a recessive
          epistasis example where a BBEE generation is crossed with a bbee generation, the F2 ratios are 9:3:4 (9 parts B-E-), 3 parts bbE-, and 4 parts --ee).

          A situation where one dominant allele is sufficient to mask the effect of some other allele is known as dominant epistasis. In a dominant
          epistasis example where a BBEE generation is crossed with a bbee generation, the F2 ratios are 12:3:1 (12 parts B---), 3 parts bbE-, and 1 part bbee).
          Should a bbee phenotype be the same as a B--- phenotype, this ratio can be simplified to 13:3 - this case is known as dominant suppression.

        \subsubsection{Redundant Genes}
          Redundant genes are two genes that control a very similar phenotype. As an example, if genes A and B are redundant and a AABB generation is crossed
          with a aabb generation, the F2 ratios are 15:1.

        \subsubsection{Heterogeneous Traits and Complementation}
          Phenotypic traits that can arise as a result from multiple different genes are known as heterogeneous. Thus, it is possible that two individuals
          sharing the same traits have differing genetic causes of those traits. If these individuals have offspring which exhibit a wild-type phenotype, then
          complementation has occurred, meaning that different genes controlled the mutant phenotype for both parents, and that they were both recessive. Thus,
          each parent could apply a dominant allele to complement the other parent's recessive mutant allele.

        \subsubsection{Penetrance and Expressivity}
          A phenotype may depend on more factors than just the underlying genotype. These factors may include environmental factors, modifier genes, and random
          chance. Penetrance is used to describe how many members of a population with a particular genotype show the expected phenotype. Penetrance can be complete
          (i.e. 100\%) or incomplete. Expressivity refers to the intensity with which a particular genotype is expressed in a phenotype, which may be variable or
          unvarying.

        \subsubsection{Modifier Genes}
          Genes that alter the phenotype produced by alleles of other genes are known as modifier genes.

\end{document}
