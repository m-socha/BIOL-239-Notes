\documentclass[12pt,titlepage]{article}
\usepackage[margin=1in]{geometry}
\usepackage{parskip}

\usepackage{hyperref}
\hypersetup{
  linktoc=all
}

\begin{document}
  \begin{titlepage}
    \vspace*{\fill}
    \centering

    \textbf{\Huge BIOL 239 Course Notes} \\ [0.4em]
    \textbf{\Large Genetics} \\ [1em]
    \textbf{\Large Michael Socha} \\ [1em]
    \textbf{\large 4A Software Engineering} \\
    \textbf{\large University of Waterloo} \\
    \textbf{\large Spring 2018} \\
    \vspace*{\fill}
  \end{titlepage}

  \newpage 

  \pagenumbering{roman}

  \tableofcontents

  \newpage

  \pagenumbering{arabic}

  \section{Course Overview}
    \subsection{Logistics}
      \begin{itemize}
        \item \textbf{Professor:} Christine Dupont
      \end{itemize}

    \subsection{Overview of Topics}
      Topics covered in this course include:
      \begin{itemize}
        \item Transmission of traits
        \item Gene structure, function, and transmission
        \item Genes as carriers of information
        \item Gene regulation
        \item Population genetics
      \end{itemize}

  \newpage

  \section{Mendelian Genetics}
    \subsection{Artificial Selection}
      Artificial selection is the process in which humans select plants and animals to breed based 
      on particular traits. Artificial selection has been performed for tens of thousands of years,
      long before humans had a solid understanding of genetics. Examples of plants and animals domesticated
      through selective breeding include dogs, goats, sheep, cattle, cabbage, and maize.

    \subsection{Key Definitions}
      \begin{itemize}
        \item \textbf{Phenotype:} An observable characteristic in an individual, which can include an appearance or behaviour
        \item \textbf{Genotype:} The genetic composition of an individual, which in turn controls phenotypes
        \item \textbf{Gene:} A discreet unit of heredity
        \item \textbf{Allele:} An alternative form of a single gene
        \item \textbf{Dominant Allele:} An allele that is manifested in a phenotype regardless of the other allele.
        The phenotype in which this allele is manifested is known as the dominant phenotype.
        \item \textbf{Recessive Allele:} An allele that does not have an effect on a phenotype in when a dominant allele is present.
        The phenotype in which this allele is manifested is known as the recessive phenotype.
        \item \textbf{Parental Types:} Phenotypes that reflect a previously existing parental combination
        \item \textbf{Recombination Types:} Phenotypes that reflect a new combination of genes that occurs during gamete formation
        \item \textbf{Filal Generation:} Subsequent offspring in a sequence of breeding, starting with the parental (P) generation
      \end{itemize}

    \subsection{Mendel's Experiments}
      Mendel was a 19th century Augustinian monk that ran a series of experiments regarding trait inheritance. Much of the value of
      Mendel's experiments was due to how they featured:
      \begin{itemize}
        \item Using garden peas, which are easy to cross-fertizile, produce a large number of offspring, and have a short growing season
          (reproduce frequently). Garden peas also have clear-cut forms of the traits to test (e.g. yellow vs green, round vs wrinkled).
        \item Establishment of pure-breeding lines, which are lines are only breed with other members of the same genotype
        \item Carefully controlled breeding (use of reciprocal crosses and self-fertilization)
      \end{itemize}

    \subsection{Genetics Prior to Mendel}
      Multiple incorrect theories existed prior to Mendel's laws of inheritance. These included:
      \begin{itemize}
        \item One parent contributing more to an offspring the other - disproved through reciprocal crosses
        \item Parental traits become blended (i.e. no discrete traits) - disproved through reappearance of reciprocal traits
      \end{itemize}

    \subsection{Mendel's Laws}
      Through his experimentation, Mendel devised the following laws of inheritance:

      \subsubsection{Law of Segregation}
        This law states that each trait is controlled by two alleles which separate upon gamete formation and recombine when a zygote is
        formed.

      \subsubsection{Law of Independent Assortment}
        This law states that alleles for separate phenotypic traits are transmitted to offspring independent of one another. This can be demonstrated
        through dihybrid experiments, where individuals differing in multiple traits are crossed with one another.

      \subsubsection{Law of Dominance}
        This law introduces the concept of dominant and recessive alleles, and states that recessive alleles do not affect the phenotype unless
        both alleles are recessive.

    \subsection{Types of Crosses}
      \subsubsection{Monohybrid crosses}
        Monohybrid crosses are mating between homozygous (i.e. both alleles the same) individuals that only differ in one trait.
        The F1 filal features only the dominant phenotype, while the F2 filal features the dominate phenotype and recessive phenotype in a 3:1 ratio.

      \subsubsection{Dihybrid crosses}
        Dihybrid crosses are mating between homozygous individuals that only differ in two unrelated traits. The F1 filal features both dominant phenotypes, while
        the F2 filal features a 9:3:3:1 ratio of traits 1 and 2 both being dominant, trait 1 being dominant and trait 2 being recessive,
        trait 1 being recessive and trait 2 being dominant, and both traits being recessive.

      \subsubsection{Multihybrid crosses}
        Dihybrid crosses are mating between individuals that only differ in three or more unrelated traits.

    \subsection{Analysis Techniques}
      \subsubsection{Punnett Squares}
        A Punnett Square is a chart that lists the possible gametes that may combine in a cross (one gamete list forms 1 dimension), and lists all
        possible resulting combinations.

      \subsubsection{Branched-line Diagrams}
        A branched-line diagram features one column for each gene in a cross, with the resulting phenotypes shown at the end. This type of diagram
        is useful for multihybrid crosses, where Punnett Squares become hard to read.

      \subsubsection{Probability Rules}
        Probability rules (i.e. product rule and sum rule) can be also be used to analyze the results of crosses, often in combination with the above
        techniques.

  \newpage

  \section{Extensions to Mendel for Single-Gene Inheritance}
    \subsection{Incomplete Dominance}
      Incomplete dominance describes the situation where F1 offspring have a phenotype that resembles neither of their parents, but rather appears as a blend
      of the two. An example is crossing red and white flowers to generate the F1 generation, and the F1 generation's flowers being pink.

    \subsection{Codominance}
      Codominance describes the situation where F1 offspring have a phenotype that combines features from both parents. An example would be crossing dotted and
      spotted lentils to generate the F1 generation, and the F1 generation being both spotted and dotted.

    \subsection{Genes with more than 2 Alleles}
      Some genes can have more than 2 alleles. In this case, filal genotype and phenotype ratios can be much different from those experiments with only
      2 alleles.

    \subsection{Gene Mutations}
      Mutations of genes occur in nature at a fairly low frequency; the frequency of gametes with a mutation for a particular gene is typically between
      1 in 10000 to 1 in 1000000, depending on the gene. Such mutations are allow for the creation of new alleles.

    \subsection{Allele Frequencies}
      The percentage of a particular allele in a gene's entire population is known as its allele frequency. Alleles that have a relatively high frequency
      (typically over 1\%) are known as wild-type alleles, while other alleles are known as mutant alleles. A gene with only one wild-type allele is known
      as monomorphic, while genes with multiple wild-type alleles are known as polymorphic.

    \subsection{Pleiotropy}
      Pleiotropy is the concept of a single gene determining multiple (often seemingly unrelated) phenotypic traits. Mendel himself observed this phenomenon
      during his experiments, where pea seed coat color matched to flower color, implying that there was a common control for both traits. Note that a single
      allele may be dominant with respect to some traits and recessive with respect to others, so dominant and recessive for such alleles must be defined
      with the context of a specific phenotype.

    \subsection{Recessive Lethal Alleles}
      Some alleles that are harmless when combined with a different allele can render an individual inviable when there are two copies of that allele. This
      allele is recessive in terms of lethality, but may be dominant with respect to some other phenotype. In this case, the phenotypic ratios of this dominant
      trait in the F1 generation would be 2:1 instead of 3:1, since individuals with two copies of this trait's dominant allele cannot survive. Note that some
      recessive lethal alleles may cause delayed lethality, in which case the phenotypic ratios would not be 2:1, since affected individuals may survive for
      some time (and may even reproduce).

  \newpage

  \section{Extensions to Mendel for Gene Interactions}
    Traits in an organism that arise from the actions of multiple genes are known as polygenic, while traits that are controlled by multiple genes and the environment
    are known as multifactorial. These two types of traits are knows as complex traits, and account for the majority of traits in organisms. Genotypic classes are
    groupings of related genotypes that produce a particular phenotype.

    \subsection{Complementary Gene Action}
      Complementary gene action refers to multiple genes working together to produce a particular trait. As an example, consider a combination of 2 genes with alleles
      Aa and Bb, and assume that a certain trait only surfaces in the A-B- phenotype. In the F2 population, this will result in a 9:7 dominant to non-dominant trait
      ratio.

    \subsection{Epistasis}
      Epistasis refers to one allele masking the effect of another allele. The allele performing the masking is known as the epistatic allele.

      A situation where two recessive alleles are required to mask the effect of some other allele is known as recessive epistasis. In a recessive
      epistasis example where the b allele may be epistatic and a BBEE generation is crossed with a bbee generation, the F2 ratios are 9:3:4 (9 parts B-E-),
      3 parts bbE-, and 4 parts --ee).

      A situation where one dominant allele is sufficient to mask the effect of some other allele is known as dominant epistasis. In a dominant
      epistasis example where the B allele may be epistatic and a BBEE generation is crossed with a bbee generation, the F2 ratios are 12:3:1 (12 parts B---),
      3 parts bbE-, and 1 part bbee). Should a bbee phenotype be the same as a B--- phenotype, this ratio can be simplified to 13:3 - this case is known as
      dominant suppression.

    \subsection{Redundant Genes}
      Redundant genes are two genes that control a very similar phenotype. As an example, if genes A and B are redundant and a AABB generation is crossed
      with a aabb generation, the F2 ratios are 15:1.

    \subsection{Heterogeneous Traits and Complementation}
      Phenotypic traits that can arise as a result of multiple different genes are known as heterogeneous. Thus, it is possible that two individuals
      sharing the same traits have differing genetic causes of those traits. If these individuals have offspring which exhibit a wild-type phenotype, then
      complementation has occurred, meaning that different genes controlled the mutant phenotype for both parents, and that they were both recessive. Thus,
      each parent could apply a dominant allele to complement the other parent's recessive mutant allele.

    \subsection{Penetrance and Expressivity}
      A phenotype may depend on more factors than just the underlying genotype. These factors may include environmental factors, modifier genes, and random
      chance. Penetrance is used to describe how many members of a population with a particular genotype show the expected phenotype. Penetrance can be complete
      (i.e. 100\%) or incomplete. Expressivity refers to the intensity with which a particular genotype is expressed in a phenotype, which may be variable or
      unvarying.

    \subsection{Modifier Genes}
      Genes that alter the phenotype produced by alleles of other genes are known as modifier genes.

  \newpage

  \section{Pedigrees}
    \subsection{Single-Gene Traits in Humans}
      Most human traits are controlled by the interaction of many genes. In fact, most single-gene traits in humans evoke life-threatening diseases, such as
      cystic fibrosis and Tay-Sachs disease.

    \subsection{Studying Hereditary Traits in Humans}
      Humans make poor genetic testing subjects; on top of many ethical issues, humans tend to not have many offspring, do not interbreed much
      (i.e. no F2 generations), and rarely choose mates on purely genetic grounds. Human genetic lineage is often analyzed using pedigrees, which
      are diagrams detailing a family's genetic features over several generations.

    \subsection{Pedigree Notation}
      \begin{itemize}
        \item \textbf{Sex:}
        \begin{itemize}
          \item \textbf{Male:} Square
          \item \textbf{Female:} Circle
          \item \textbf{Unspecified:} Diamond
        \end{itemize}

        \item \textbf{Trait infliction:}
          \begin{itemize}
            \item \textbf{Trait inflicted:} Filled shape
            \item \textbf{Trait not inflicted:} Unfilled shape
          \end{itemize}

        \item \textbf{Mating:}
          \begin{itemize}
            \item \textbf{Consanguineous (i.e. from same ancestor) mating:} Double horizontal line
            \item \textbf{Non-Consanguineous mating:} Single horizontal line
          \end{itemize}
      \end{itemize}

    \subsection{Patterns of Inheritance}
      Dominant traits tend to follow a vertical pattern of inheritance, being transferred over multiple generations. On the other hand,
      recessive traits tend to follow a horizontal pattern of inheritance, where in order for offspring to exhibit a recessive phenotype,
      both parents must carry the allele that causes that phenotype.

  \newpage

  \section{Chromosomes}
    \subsection{Experimentation History}
      In the 17th century, semen was determined to contain spermatozoa, which was hypothesized to be able to enter an egg and achieve fertilization.
      This hypothesis was confirmed by further experimentation in the 19th century. Around 1880, threadlike structures later confirmed to be
      chromosomes were observed during cell division. In the early 1900s, a link was established between chromosomes and the Mendelian principles of
      segregation and independent assortment, giving rise to the chromosomal theory of inheritance.

    \subsection{What are Chromosomes?}
      Chromosomes are organelles that package DNA. Most cells that compose an individual, as well as zygotes, carry two matching sets of chromosomes.
      These cells are knows as diploids. Chromosomes in a pair that contain the same genes are called homologous chromosomes (or homologues).
      Since one chromosome in each pair comes from each parent, homologues may contain different alleles. Cells that only contain one set of chromosomes,
      such as gametes, as known as haploids.

      A karyotype is a visual description of the complete set of chromosomes found in the cell of an organism. In a karyotype, homologous pairs are shown
      together ordered by decreasing size. The matching pairs of chromosomes are known as autosomes, which are distinct from sex chromosomes.

    \subsection{Sex Chromosomes and Sex Determination}
      Sex chromosomes determine the sex of an organism. In humans, the 23rd pair of chromosomes are sex chromosomes, where females have an XX pair and males have
      an non-homologous XY pair. The Y chromosome's SRY (sex-determining region of Y) gene is specifically what determines maleness. Sex-determining mechanisms
      can vary significantly between different species.

  \newpage

  \section{Mitosis}
    Mitosis is a process of cell division in which a cell produces two genetically identical daughter cells. In the cell cycle, mitosis is known as the M phase.

    \subsection{Interphase}
      Interphase is the part of the cell cycle that occurs between divisions. Interphase consists of the gap 1 (G1), synthesis (S), and gap 2 (G2) phases.

      \subsubsection{G1}
        G1 lasts from the birth of a new cell until chromosome replication. In this phase, most of a cell's growth occurs. The duration of G1 can vary greatly,
        from between a few hours in a newly developing embryo, to the remaining of an individual's lifetime for mature brain cells.

      \subsubsection{S}
        In the S phase, the cell duplicates its genetic material by synthesizing DNA. Each chromosome doubles to produce identical sister chromatids, which become
        visible once chromosomes condense at the beginning of mitosis. These sister chromatids remain joined to each other at the centromere until mitosis begins.

      \subsubsection{G2}
        During G2, the cell may grow, though typically less than during G1. The cell also synthesizes proteins necessary for mitosis.

      Also, during interphase, fine microtubules known as centrioles radiate out into a cell's cytoplasm from a structure known as a centrosome.

    \subsection{Phases of Mitosis}

      \subsubsection{Prophase}
        During interphase, the cell nucleus remains intact, and chromosomes are masses of chromatin, which look like a mass of fine tangled string. At prophase,
        chromosomes condense, and move apart toward opposite poles of a cell. Also, nucleoli begin to disappear, and microtubules disappear and are replaced by a
        set of dynamic microtubules that grow from and shrink toward their centrosomal organizing centers. The centrosomes also move apart to opposite poles of the
        cell.

      \subsubsection{Prometeaphase}
        In prometaphase, the nuclear envelope breaks down, and microtubules from the centrosomes invade the nuclues. Sister chromatids attach to microtubules from
        opposite centrosomes through the kinetochore, a structure in the centromere region of each chromatid specialized for transport.

      \subsubsection{Metaphase}
        In metaphase, chromosomes align on the metaphase plate, which is an equator halfway between the cell's two poles. In this phase, forces pulling sister
        chromatids toward opposite poles are in balance with tension across the chromosomes.

      \subsubsection{Anaphase}
        During anaphase, the centromeric connections are severed, and separated sister chromatids begin to move to opposite cell poles to which they are connected
        through kinetochere microtubles. This separation of sister chromatids to pass genetically identical material onto both daughter cells is known as disjunction.

      \subsubsection{Telophase}
        During telophase, the nuclear membrane and nucleoi reform, the spindle fibres disappear, and chromosomes untangle to become a mass of chromatin. This phase can be
        seen as somewhat of a reverse of prophase.

      \subsubsection{Cytokinesis}
        In cytokenesis, the nuclei emerging at each end of the elongated parent cell as packaged into two separate daughter cells. Processes of cytokenesis may actually
        begin during anaphase, though do not complete until after telophase.

  \newpage

  \section{Meiosis}
    The majority of cells that make up an individual are somatic cells, which either mitotically divide or are arrested in a G phase of the cell cycle. On the other hand,
    Germ cells are incorporated into reproductive organs, where they eventually undergo meiosis, a type of cell division that yields 4 gamete haploid cells from a single
    diploid cell. Meiosis contains two separate cell divisions, known as meiosis I and meiosis II.

    \subsection{Meiosis I}

      \subsubsection{Prophase I}
        Lepotene is the first substage of prophase I, and is the phase in which chromosomes start to thicken. These chromosomes have already been priorly duplicated.

        In Zygotene, each chromosomes seeks out its homologous partner, which become zipped together in a process known as synapsis. Each synapses chromosome pair is known
        as a bivalent (contains two chromosomes), or a tetrad (contains four chromatin). One side of a bivalent has maternal genetic material, and the other has paternal
        genetic material.

        Pachytene begins after synapsis, and consists of structured called recombination nodules exchanging parts of maternal and paternal chromatids of homologous chromosomes.
        This exchange is known as recombination or crossing-over.

        In diplotene, the synaptonemal zipper complex begins to disappear, and the tetrads of four chromatin become visible and start to move apart. However, homologous
        chromosomes remain tightly connected at areas called chiasmata, which represent sites where recombination occurred.

        In diakinesis, the chromatids condense and shorten even further. Also, the nuclear membrane begins to break down, and spindles begin to attach to kinetochere structures.

      \subsubsection{Metaphase I}
        During metaphase I, the tetrads formed in prophase I line up along the metaphase plate. These tetrads are attached to spindles on opposite poles, and begin to separate.

      \subsubsection{Anaphase I}
        In anaphase I, the chiasmata joining homologues are dissolved, allowing the chromosomes to separate and move toward opposite poles.

      \subsubsection{Telophase I}
        Telophase I begins when nuclear membranes start to form around the chromosomes that have moved to each pole. Each daughter nuclei contains half the number of chromosomes
        in the original parent (meiosis I is known as reducational division), but there are two sister chromatids for each chromosome.

    \subsection{Meiosis II}
      Following a brief interphase following meiosis I known as interkinesis, Meiosis II is performed on both daughter cells. Meiosis II is a very similar process to mitosis,
      and yields four haploid cells in total.

    \subsection{Mistakes in Meiosis}
      Segregational errors during meiotic divisions can lead to aberrations. An example cause is when homologues fail to separate during meiosis I, causing a trisomy through
      a process known as nondisjuction. Many trisomies are lethal in utero, though others, such as trisomy 21 (i.e. Down syndrome), are not.

      On a separate note, some hybrid animals carry non-homologous chromosomes that cannot pair up and segregate during meiosis. Viable gametes cannot form in such animals,
      so they are sterile.

    \subsection{Meiosis and Genetic Diversity}
      Two main factors contribute to genetic diversity in meiosis. The first is that the pole maternal or paternal homologues migrate to is random, so the homologue pairs
      are able to assort independently of one another. The second factor is the resuffling of genetic material that occurs during recombination.

  \newpage

  \section{Gametogenesis}
    Embryonic germ cells (collectively known as a germ line) are specialized diploid cells that can give rise to haploid cells through meiosis. These meiotic divisions lead
    to the production of gametes, with either the resulting haploids being gametes, or some of the resulting haploids undergoing some sort of differentiation to become
    gametes.

    \subsection{Oogenesis}
      Oogenesis is the process that produces egg cells, also known as ovums. Oocytes are cells that undergo meiotic divisions to ultimately generate on ovum. About half
      a million primary oocytes are stored in the ovaries, and are arrested in the diplotene substage of prophase I. During ovulation, a primary oocyte is released and
      produces a secondary oocyte and a far smaller polar body. The first polar body usually arrests its development, while the secondary oocyte proceeds as far as the
      metaphase of meiosis II. Upon fertilization, meiosis II completes, with the secondary oocyte producing a mature ovum and a second polar body. The long interval
      between the beginning and completion of meiosis for oocytes may lead to segregational errors such as trisomies for older females.

    \subsection{Spermatogenesis}
      Spermatogenesis is the process that produces sperm cells. Spermatocytes are cells that undergo meiotic divisions to produce sperm cells. Unlike oocytes,
      spermatocytes undergo a symmetric meiosis I and II, yielding four spermatids for each spermatocyte. Spermatogenesis can continue throughout a male's lifetime.

  \newpage

  \section{Validation of Chromosome Theory}

    \subsection{Key Claims}
      The key claims of the chromosomal theory of inheritance are that:
      \begin{enumerate}
        \item Hereditary information is stored on genes, which are located in chromosomes.
        \item The egg and sperm contribute equally to the genetic composition of the resulting zygote.
      \end{enumerate}

    \subsection{Overview of Evidence}
      Based on already covered material, the following points support the chromosomal theory of inheritance:
      \begin{enumerate}
        \item The number of chromosomes in somatic cells is constant.
        \item Sexual identity is associated with particular chromosomes.
      \end{enumerate}

      The following evidence would strengthen the theory's claims:
      \begin{enumerate}
        \item The inheritance of genes corresponding to the inheritance of chromosomes.
        \item The transmission of particular chromosomes corresponding with the transmission of particular traits (i.e. traits other than sex).
      \end{enumerate}

      \subsection{Experimental Results}
        Key experiments were run by American biologist Thomas Hunt Morgan on a type of fly called Drosophilia. These experiments ultimately
        confirmed that a gene controlling eye color resides on an X chromosome, which is also used for sex determination (this makes the eye
        color an X-linked trait). These results were obtained by observing the phenotypic ratios of eye color between males and females, as
        well as analyzing individuals that had undergone non-disjunction to confirm that eye color was ultimately controlled by genes on X
        chromosomes rather than an individual's sex.

  \newpage

  \section{Aneuploidy}
    Aneuploidy refers to an abnormal number of chromosomes in a cell (i.e. not an exact multiple of haploid number $n$). Individuals lacking a
    chromosome in a pair are known as monosomic, while individuals with an extra chromosome are known as trisomic.

    \subsection{Autosomal Aneuploidy}
      Autosomal aneuploidy tends to be lethal for most chromosomes. However, there are some exceptions, such as trisomy 21 in humans, which
      causes Down syndrome. As a general rule, autosomal trisomies can sometimes be viable, while autosomal monosomies cannot.

    \subsection{Sex Chromosome Aneuploidy}
      Aneuploidies in sex chromosomes tend to be more viable than in autosomal chromosomes. For aneuploidies of the X chromosome in particular,
      X-chromosome inactivation represses the expression of most genes on all but one X chromosome. However, X-chromosome aneuploidies commonly
      cause infertility, since X-chromosome reactivation occurs in the oogonia, which is normally necessary in order to ensure that all of
      the produced ovums can support development after fertilization (as opposed to half with only X-chromosome inactivation). X-chromosome
      reactivation also occurs in males with multiple X chromosomes, typically preventing them from producing sperm.

    \subsection{Meiotic Nondisjunction}
      Mistakes in chromosome segregation during meiosis can cause aneuploidies. Nondisjunction during meiosis I produces four cells with
      aneuploidies, while nondisjunction during meiosis II produces only two such cells. Studies suggest that meiotic nondisjunction is may
      result from problems during recombination. In particular, a lack of chiasmata formation removes the mechanism for ensuring that
      maternal and paternal chromosomes go to opposite poles during anaphase I.

    \subsection{Mitotic Nondisjunction}
      Aneuploidies may also result from the failure of sister chromatids to separate during mitosis, producing trisomic and monosomic daughter
      cells. Other mistakes may also occur, such as chromosome loss, which produces one monosomic and one diploid cell. Aneuploid cells may
      survive and undergo further rounds of mitosis, potentially producing a mosaic organism with differing chromosome numbers in across
      different tissue. One such example is with gynandromorphic organisms, which are organisms that contain both male and female characteristics.
      Another example is with Turner syndrome (XO); most affected individuals start off as XX zygotes, but lose an X chromosome early during embryonic
      development.

  \newpage

  \section{Euploidy}
    Euploid cells contain only complete sets of chromosomes, but might be monoploids (containing only 1 set of chromosomes) or polyploids
    (containing 3 or more sets of chromosomes).

    \subsection{Monoploidy}
      Monoploid organisms tend to develop by parthenogenesis, where an ovum develops without having been fertilized. Due to the mechanics of meiosis,
      monoploids are typically infertile. One potential benefit of monoploidy, often used by plant breeders, is that desirable recessive traits appear
      since there is no matching (potentially dominant) allele.

    \subsection{Triploidy}
      Triploids, which have three sets of chromosomes, can result from the union of gamete and diploid cells, and are almost always sterile.
      This sterility is often used by plant breeders, who may want to deliberately grow seedless fruit. This characteristic of sterility can
      be extended in general to polyploids with odd numbers of chromosome sets, since they cannot produce balanced gametes.

    \subsection{Tetraploidy}
      Tetraploids have four complete sets of chromosomes. Tetraploids can arise from the nondisjunction of all chromosomes during mitosis. Unlike triploids,
      tetraploids can be fertile, since they can produce balanced (diploid) gametes. This possibility of fertility can be extended in general to polyploids
      with even numbers of chromosome sets, since they may be able to produce balanced gametes.

  \newpage

  \section{Gene Linkage and Recombination}
    Some genes do not follow Mendel's law of independent assortment, but tend to travel together, exhibiting a behavior known as gene linkage. In such
    genes, parental classes of alleles tend to appear in offspring much more frequently than recombinant classes.

    A common cause for gene linkage is that linked genes are on the same chromosome, in which case they are considered syntetic. These genes do not always
    travel together due to recombination, but tend to travel together in most cases, especially when they are close together on the same chromosome.

    \subsection{Testcrosses for Linkage Detection}
      Two-point (i.e. ``two gene'') testcrosses can be carried out to detect gene linkage. These tests work by breeding a parent that only contributes
      recessive alleles with a parent that is dihybrid for the traits to test. Should the resulting phenotypes not be in a 1:1:1:1 ratio, then gene linkage
      is likely occurring.

    \subsection{Observing Recombination}
      Recombination a key reason why genes on the same chromosome do not always travel together. This can be confirmed by experiments by observing
      chiasmata, which form where gene recombination occurs.

    \subsection{Recombination Frequencies}
      Recombination Frequencies (RFs) concerning two genes can be used as an indicator for the distance between two genes on a chromosome. Units of measure
      along a chromosome include RF percentage points, centimorgans (cM) and map units (m.u.), where 1\% RF = 1cM = 1m.u. Recombination frequencies tend to
      be 50\% for genes on different chromosomes and can approach 50\% for genes on the far apart on the same chromosome, but never exceed 50\%.

    \subsection{Linkage Analysis}
      Determining whether two genes are linked required some statistical analysis. It is not sufficient to say that the genes did not assort in a perfect
      1:1:1:1 ratio, since real-world experimentation, especially on small sample sizes, can produce deviations from theoretically expected values.

      \subsubsection{Chi-Square Test}
        A chi-square test can be used to determine the probability that a certain gene arrangement could have occurred by chance, assuming an expected
        1:1:1:1 ratio (i.e. assume the linkage hypothesis is false (null hypothesis), and then determine whether this assumption matches the data).

        $$\chi^2 = \sum\frac{(n_{obs} - n_{exp})^2}{n_{exp}}$$

        The degrees of freedom (df) must also be determined, which is 1 less than the number of observed classes. For example, if the experiment contains
        parental and recombinant types, then the df is 1. Based on the df and the $\chi^2$ value, the p-value (probability) of the null hypothesis can be
        determined. Generally speaking, p-values below 0.05 are considered a boundary between significance and non-significance; null hypotheses below 0.05
        are typically rejected.

  \newpage

  \section{Gene Mapping}
    Gene mapping can be used to find the positions of genes on locations (i.e. loci) on chromosomes. These maps can be used to predict inheritance patterns
    by estimating recombination frequencies.

    \subsection{Two-Point Crosses}
      Two-point crosses involve determining the relative positions of two genes on a chromosome. While fairly simple to analyze, two-point crosses have many
      limitations, including that it may be hard to determine gene order for genes close together, and that actual map distances do not always add up accurately
      for genes far apart. For genes far apart, it is better to perform a series of two-point crosses using genes that lie between the two genes being analyzed,
      or to perform a three-point cross.

    \subsection{Three-Point Crosses}
      Three-point crosses involve determining the relative positions of three genes on a chromosome. The main advantage of three-point crosses over two-point
      crosses are that they allow for correcting mapping distance discrepancies caused by double crossovers. In the resulting progency, parental classes
      will be the most frequent resulting genotypes. The next most frequent genotypes will result from single crossover recombinations between the two most
      distant genes (known as region I). The most frequent after that will result from single crossover recombinations between the two closest adjacent
      genes (known as region II). The two genotypes with the lowest frequency result from double crossovers. These double crossovers must be counted twice
      when calculating distances between the two outside genes.

      Note that although three-point crosses are able to accurately estimate mapping distances between the second (i.e. middle) gene with the first and third gene,
      mapping distances between the first and third gene tend to be incorrect.

  \newpage

  \section{DNA}

    \subsection{Evidence DNA is Genetic Material}
      In 1869, a substance later confirmed to be DNA (deoxyribonucleic acid) was extracted from human cells. A further experiment carried out in 1923, which
      used a Schiff reagent to stain DNA red, showed that DNA is stored almost exclusively in chromosomes.

      Even though DNA was confirmed to exist in chromosomes, this was not strong evidence that it had anything to do with genes. Certain proteins stored in
      chromosomes, which can be much more complex than DNA, were considered by many scientists to be the building blocks of genes.

      In 1923, Frederick Griffith carried out an experiment that showed that genetic material from dead bacteria cells could somehow be transmitted to live cells. 
      This ability of a substance to change the genetic characteristics of another organism is known as transformation. These results were published in 1928, and
      Oswald Avery subsequently led a team that published findings in 1944 that this transforming principle was DNA.

      Further evidence that DNA is genetic material was found in 1952, where Alfred Hershey and Martha Chase determined that viruses called phages infect bacteria
      by injecting DNA into cells.

    \subsection{Structure of DNA}
      DNA is a large polymer built from subunits known as nucleotides. Each nucleotide is composed of a deoxyribose sugar, a phosphate, and one of four nitrogenous
      bases (Adenine (A), Guanine (G), Thymine (T), and Cytosine (C)). A nucleoside consists of a base with a covalent attachment to the deoxyribose sugar, and the
      addition of a phosphate forms a nucleotide.

      The double helix structure of DNA was first proposed by James Watson and Francis Crick in a 1953 paper. A piece of evidence for this theory is X-ray diffraction
      data, which showed that DNA is helical. Another piece of evidence is that A:T and C:G exist in nearly perfect 1:1 ratios, which occurs because two anti-parallel
      strands of DNA are held together by hydrogen bonding between these base pairs.

      The DNA helix completes a full turn every 34 angstroms (an angstrom is $10^{-10}$m), the diameter of the molecule is 20 angstroms, and the spacing between repeating
      units is 4.3 angstroms.

      \subsubsection{Alternative Forms of DNA}
        DNA molecules in a standard double-helix right spiral formation are known as B-form DNA. An alternative form of DNA is known as Z-form, where the helix spirals
        to the left and the backbone has a zigzagging shape. While there is evidence that Z-form DNA may exist in living organisms, no clear biological role has yet
        been determined.

        Eukaryotic nuclear chromosomes are all long, linear double helixes. However, in some instances, such as with chromosomes of prokaryotic bacteria or the chromosomes
        of some organelles such as mitochondria, chromosomes can be circular.

        Some viruses carry only a single stand of DNA, with can serve as a template for making a second strand once inside a host. Some viruses use RNA as their genetic
        material, which contains the sugar ribose instead of deoxyribose, uracil (U) as a base instead of thymine (T), and tends to be single-stranded and short.

    \subsection{DNA Replication}
      \subsubsection{Models of Replication}
        The three main models of replication are:
        \begin{itemize}
          \item \textbf{Conservative:} One of two resulting helixes consists only of original DNA strands, while the other consists of newly synthesized strands.
          \item \textbf{Semiconservative:} One strand of each resulting helix comes from the parent, other is synthesized.
          \item \textbf{Dispersive:} Both resulting helixes carry stands of original DNA interspersed with newly synthesized DNA.
        \end{itemize}

        The Watson-Crick model of DNA replication involves semiconservative replication, which has been experimentally verified. One relevant experiment was carried
        out by Matthew Meselson and Franklin Stahl, which dealt with controlling the isotopic composition of nucleotides incorporated into daughter DNA strands. These
        isotopes varied in density, so the composition of the resulting DNA could be studied by spinning the DNA in a centrifuge, where the results were consistent with
        semiconservative replication.

      \subsubsection{Molecular Mechanism of Replication}
        DNA replication is a complex process that occurs at a specific stage in the cell cycle (S phase for eukaryotes). The process consists of two stages, namely
        initiation and elongation.

        \textbf{Initiation} consists of preparing the double-helix for complementary base pairing. At first, a small region of the double helix is unwound, exposing
        short sequences of nucleotides in what is known as the origin of replication.

        An initiator protein binds to the origin of replication, which attracts an enzyme known as DNA helicase. This enzyme catalyzes the unwinding of the double
        helix, forming what is known as a replication bubble, which is bounded by two unwound DNA strands called replication forks.

        Formation of new DNA stands depends on an enzyme called DNA polymerase III. This enzyme acts according to the following rules:
        \begin{enumerate}
          \item Only unwound and single-stranded DNA can be copied.
          \item Nucleotides can only be added to the end of a chain.
          \item The DNA strand can only be built in a single direction (i.e. from 5' to 3').
        \end{enumerate}
        To start off the chain, a short strand of RNA consisting of a few nucleotides is used as a primer, off of which a DNA strand can be built. This primer is
        synthesized by an enzyme known as primase.

        \textbf{Elongation} consists of connecting nucleotides into a continuous strand of DNA. Once base-pairing nucleotides are formed from the unwound strands,
        they form a chain by the formation of phosphodiester bonds in a process known as polymerization. This polymerization is catalyzed by DNA polymerase III.

        Since polymerization only takes place in a 5' to 3' direction, the two strands forming the replication bubble will be replicated through a slightly different
        mechanism. One strand is built in the direction the helix unwinds, which is known as the leading strand. The other strand's 5'-to-3' direction runs in the
        opposite direction of the strand unwinding, and is known as the lagging strand. The lagging strand is replicated using around 1000 small strands known as
        Okazaki fragments, which are later merged by an enzyme known as DNA polymerase I that replaces the many RNA primers with DNA.

      \subsubsection{Ensuring Integrity of Genetic Information}
        Since DNA is the sole repository for a vast amount of information concerning how a cell functions, the integrity of its information is of utmost importance.
        Mechanisms for ensuring integrity include:
        \begin{itemize}
          \item \textbf{Redundancy:} One strand of double-stranded DNA specifies the sequence of the other, which provides a mechanism for detecting errors.
          \item \textbf{Precision of cellular replication machinery:} Errors during replication are very rare in the first place.
          \item \textbf{Enzymes that repair chemical damage to DNA:} Cells contain enzymes that can repair various types of damage to DNA.
        \end{itemize}

      \subsubsection{Replication of Circular Chromosome}
        A replication bubble in circular chromosomes may cause excessive torsion between intertwined strands as it expands. Topiosomerase mitigates this issue by
        unwinding DNA strands and separating the generated molecules.

      \subsubsection{Replication of Long Chromosomes}
        DNA replication in eukaryotes would take too long if it only involved a single replication bubble. Instead, DNA replication occurs at many points along
        a chromosomes, forming replicons that are later joined.

      \subsubsection{Replication of Linear Chromosomes}
        DNA replication only occurs in a single direction (i.e. from 5' to 3'). Thus, there can be issues in replicating DNA at the 5' ends of chromosomes.
        This issue is mitigated by telomeres, which consist of repeating sequences of DNA on the end of DNA molecules that do not encode proteins. A
        ribonucleoprotein called telomerase can add additional telomore sequences to the end of a DNA molecule, delaying eventual cell death. However, most
        somatic cells do not express telomerase, so their number of replications is limited. Increased levels of telomerase in somatic cells are known to increase
        the risk of cancer, so adding telomerase is not a simple ``fountain of youth''.

  \newpage

  \section{Genetic Code}
    Gene expression refers to the flow of genetic information from DNA or RNA to polypeptides (chains of amino acids). Like nucleotides are the building blocks
    of DNA, amino acids are the building blocks of proteins. There are 20 amino acids that are encoded by DNA or RNA; rarer amino acids undergo further modification
    after synthesis. Amino acids are encoded by nucleotide triplets called codons. There are 64 possible codons in total, 61 of which represent the 20 amino acids,
    and 3 of which signify stops.

    \subsection{``Cracking'' the Genetic Code}

      \subsubsection{Yanofsky's Experiments}
        In the 1960s, Charles Yanofsky determined the following:
        \begin{itemize}
          \item \textbf{Nucleotide sequences are colinear with amino acid sequences.} The evidence is that changing the order of DNA mutations mapped to positions
            of amino acid substitutions.
          \item \textbf{Codons are composed of multiple nucleotides.} The evidence is that altering different nucleotide pairs may affect the same amino acid.
          \item \textbf{Each nucleotide is only part of one codon.} The evidence is that point mutations of a single nucleotide pair affect only a single amino acid.
        \end{itemize}

      \subsubsection{Triplet Codons}
        In research published in 1961, Francis Crick and Sydney Brenner provided evidence that DNA was composed of trios of nucleotides with a single starting point
        called reading frames. Changes that alter the grouping of nucleotides into codons are known as frame-shift mutations, and almost always abolish the function
        of the generated polypeptide. Crick and Brenner hypothesized demonstrated that combining three such insertion or deletion mutations might not result in such
        mutations. These mutations can perform what is known as intragenic suppression, which is the restoration of gene function by one mutation canceling the other
        in the same gene. The relative commonality of intragenic suppression implies that most amino acids are specified by multiple codons.

      \subsubsection{Mapping Codons to Amino Acids}
        In the 1950s, researchers detected that although most DNA is stored in the cell nucleus, protein synthesis takes place in a cell's cytoplasm. This implies that
        some intermediate DNA molecule must transport the genetic information into the cytoplasm, later shown to be messenger RNA (mRNA). Knowledge of mRNA helped lead
        to two breakthroughs that helped crack the genetic code. The first was that mRNA combined with certain other substances could be used to synthesize polypeptides
        in test tubes (in vitro). The second was development of techniques enabling the synthesis of some strands of artificial mRNA.

    \subsection{Nonsense Codons}
      There are three different triplets (UAA, UAG, and UGA) that serve as stop (or nonsense) codons. A mutation that changes an amino-acid producing codon into a stop
      codon is known as a nonsense mutation.

    \subsection{Universality}
      The genetic code is nearly universal, with genetic information often interchangeable between different species. Although there are a handful of exceptions, codon
      encoding of polypeptides is almost always the same.

\end{document}
